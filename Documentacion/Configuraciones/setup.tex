% !TEX root = my-thesis.tex


% **************************************************
% Files' Character Encoding
% **************************************************
\PassOptionsToPackage{utf8}{inputenc}
\usepackage{inputenc}



% **************************************************
% Information and Commands for Reuse
% **************************************************
\newcommand{\thesisTitle}{Desarrollo y control de una gimbal de dos grados de libertad mediante visión artificial para el seguimiento de objetivos}
\newcommand{\thesisName}{Marco Antonio Aguilar Gallardo}
\newcommand{\thesisSubject}{Innovación educativa para el desarrollo de México}
\newcommand{\thesisDate}{\today}
\newcommand{\thesisVersion}{Versión 3.0}

\newcommand{\thesisFirstReviewer}{M. en C. Antonio Flores Moreno}
\newcommand{\thesisFirstReviewerUniversity}{\protect{Asesor Tecnico}}
\newcommand{\thesisFirstReviewerDepartment}{Maestro en ciencias}

\newcommand{\thesisSecondReviewer}{Edith Hernandez Hernandez}
\newcommand{\thesisSecondReviewerUniversity}{\protect{Asesor de estilo}}
\newcommand{\thesisSecondReviewerDepartment}{Profesora}

\newcommand{\thesisFirstSupervisor}{Antonio Flores Moreno}
\newcommand{\thesisSecondSupervisor}{Maria del Carmen}

\newcommand{\thesisUniversity}{\protect{Proyecto de titulación}}
\newcommand{\thesisUniversityDepartment}{Departamento de Ingeniería}
\newcommand{\thesisUniversityInstitute}{Universidad Aeronáutica en Querétaro}
\newcommand{\thesisUniversityGroup}{Ingeniería en Electrónica y Control de Sistemas de Aeronaves}
\newcommand{\thesisUniversityCity}{Querétaro}
\newcommand{\thesisUniversityStreetAddress}{Carretera Estatal 200 Querétaro – Tequisquiapan No. 22154 }
\newcommand{\thesisUniversityPostalCode}{76278}


% **************************************************
% Debug LaTeX Information
% **************************************************
%\listfiles


% **************************************************
% Load and Configure Packages
% **************************************************
\usepackage[spanish]{babel} % babel system, adjust the language of the content
\PassOptionsToPackage{% setup clean thesis style
    figuresep=colon,%
    hangfigurecaption=false,%
    hangsection=true,%
    hangsubsection=true,%
    sansserif=false,%
    configurelistings=true,%
    colorize=full,%
    colortheme=bluemagenta,%
    configurebiblatex=true,%
    bibsys=biber,%
    bibfile=Referencias,%
    bibstyle=alphabetic,%
    bibsorting=nty,%
}{cleanthesis}
\usepackage{cleanthesis}

\hypersetup{% setup the hyperref-package options
    pdftitle={\thesisTitle},    %   - title (PDF meta)
    pdfsubject={\thesisSubject},%   - subject (PDF meta)
    pdfauthor={\thesisName},    %   - author (PDF meta)
    plainpages=false,           %   -
    colorlinks=false,           %   - colorize links?
    pdfborder={0 0 0},          %   -
    breaklinks=false,            %   - allow line break inside links
    bookmarksnumbered=true,     %
    bookmarksopen=true          %
}

% **************************************************
% Other Packages
% **************************************************
\usepackage{scrhack}
\usepackage{graphicx,lipsum,afterpage,caption}
\usepackage{biblatex}
\usepackage[nottoc]{tocbibind}
\usepackage{amssymb, amsmath, amsbsy, amsfonts}    % ECUACIONES Y SÍMBOLOS MATEMÁTICOS
\usepackage{mathrsfs}                    % para formato de letra en ecuaciones
\usepackage{amsthm}     %Libreria para agregar teoremas
\usepackage{algpseudocode}
\usepackage{algorithm}
\usepackage{listings}
\usepackage{color}
\usepackage{multicol}

% generar los cuadros de color---------------------------------------------------
\usepackage{tcolorbox}
\tcbuselibrary{listingsutf8} % o listings o minted
% Cuadro numerado para ejemplos
\newtcolorbox[auto counter, number within=section]{example}[2][]
{colback=green!5!white,colframe=green!75!black,
fonttitle=\bfseries, title=Función~\thetcbcounter: #2,#1}
% generar los cuadros de color---------------------------------------------------


\theoremstyle{definition}
\newtheorem{definition}{Definición}[section]
 
\lstset{ frame=Ltb,
framerule=0pt,
aboveskip=0.5cm,
framextopmargin=3pt,
framexbottommargin=3pt,
framexleftmargin=0.4cm,
framesep=0pt,
rulesep=.4pt,
backgroundcolor=\color{gray97},
rulesepcolor=\color{black},
%
stringstyle=\ttfamily,
showstringspaces = false,
basicstyle=\small\ttfamily,
commentstyle=\color{gray45},
keywordstyle=\bfseries,
%
numbers=left,
numbersep=15pt,
numberstyle=\tiny,
numberfirstline = false,
breaklines=true,
}

% minimizar fragmentado de listados
\lstnewenvironment{listing}[1][]
{\lstset{#1}\pagebreak[0]}{\pagebreak[0]}

\lstdefinestyle{consola}
{basicstyle=\scriptsize\bf\ttfamily,
backgroundcolor=\color{gray75},
}

\lstdefinestyle{C++}
{language=C++,
}

\definecolor{gray97}{gray}{.97}
\definecolor{gray75}{gray}{.75}
\definecolor{gray45}{gray}{.45}
