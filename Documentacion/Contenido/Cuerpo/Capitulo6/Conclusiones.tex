\chapter{Conclusiones}
En el presente trabajo se abordaron temas tanto de vision artificial como de control de sistemas, de los cuales se obtuvieron
experiencias enriquecedoras de aprendizaje y se saraon las siguientes conlusiones.
\begin{itemize}
    \item Tal y como se esperaba la implementación del algoritmo de vision artificial en ROS hizo que se pudiera tener una mejor modularidad 
    en el sistema debido a que se pudo hacer diversos nodos que podian ser ejecutados en tiempo real e incluso llegar a ser 
    ejecutados en paralelo. ROS ademas permitio tener una interfaz grafica para obtener la respuesta en el tiempo del control sin necesdiad
    de migrar datos a un programa exterior, esto permitió que al sintonizar el control uno como diseñador del mismo tenga control tenga 
    mejor noción del error y los valores de las ganacias del sistema en tiempo real.
    \item La implementación del sistea en un vehiculo aereo no tripulado requiere que el sistema emita señales de control, tal y como 
    se planteo en un inicio de este trabajo, al implementarse con ROS permite que se pueda crear diversos nodos remotos via UDP, y con 
    la tarjeta monoprocesadora ODROID en el drone, haria que la obtencion de datos en tierra sea un tarea relativamente sencilla.
    \item En el algortimo de vision artificial para el seguimiento de objetivos, una de las tareas que tomaron mas tiempo de analisis 
    fue la de clasificiación de colores, como se abordo en el capitulo 4 diversos factores ambientales afectaron a la percepción y 
    clasificiación de colores, la solución que muchos autores, citados en este trabajo, hicieron para lidiar con este problema es modificar 
    los valores del espacio de color, esto fue probado en este trabajo pero los resultados no fueron los esperados ya que debe de realizarse 
    siempre en un espacio controlado, lo que no permetiria una correcta implementación en UAV, ademas de la cantidad de procesamiento 
    que se tiene que hacer para modificar los valores cada frame, es por eso que la idea inicial de modificar los rangos de colores respecto
    al entonrno quedo descartada dando pie a la implementación del control gamma de iluminación, lo que nos permitio tener que modificar 
    solo un valor(gamma) y con esto se redujo hasta un 66\% de carga al precesador. Como se mencionó al inicio del actual trabajo se desconocia
    esta tecnica de mejorar la nitidez de colores, dando incluso mejores resultados para el usuario final, ademas pensar en la implementación
    de un sensor de brillo ahora es posible, lo cual nos ahorraria modificar el valor de gamma manualmente.
\end{itemize}

